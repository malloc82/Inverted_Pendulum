\documentclass{notes}
\usepackage[makeroom]{cancel}

\author{Ritchie Cai, Matthew Mosley \& Corey Higgins}
\title{Inverted Pendulum Summary}

\begin{document}
\maketitle 
\section{First}
For the past few weeks, we designed a system that consists of an inverted pendulum on two wheels. This design was created by using the programmable BRIC from a LEGO Mindstorms EV3 set. The overall goal of the system was to have the two wheels balance the pendulum in an upright, vertical position after a slight disturbance offsetting the inverted pendulum by a relatively small angle. (This assumed small angle comes into play in our calculations.) Additionally, we wanted the design to have a settling time after the disturbance to correct itself in less than 5 seconds.
   
The designing process began with observing and analyzing the physics behind our system. Friction was ignored as we intended our PID controller to cancel out friction's effect. As seen in our modeling document, we were able to derive the governing equations of motion behind our design. As expected, we observed that our inverted pendulum transfer function was unstable as clearly indicated by the pole existing in the positive, right half plane (seen in our pole-zero diagram). This result made a controlling mechanism a necessity in the design to gain more stability. As for the controller used, a simple PID controller was chosen along with negative feedback through the use of a LEGO Mindstorms EV3 gyro sensor.          
\end{document}